\documentclass[12pt, a4paper]{article}

\usepackage[utf8]{inputenc}
\usepackage{graphicx}

\usepackage{listings}
\usepackage[hidelinks]{hyperref}
% cosas para la biblio y tal
\usepackage[spanish]{babel}
\usepackage[babel]{csquotes}
\usepackage[backend=biber,style=apa]{biblatex}
\addbibresource{main.bib}
\DeclareLanguageMapping{spanish}{spanish-apa}

\hypersetup{
    colorlinks=true,
    linkcolor=black,
    filecolor=magenta,      
    urlcolor=blue
}

\usepackage{float}
\usepackage{import}
\usepackage{xurl}

\newcommand\fig[3]{
\begin{figure}[H]
  \centering
  \includegraphics[width=.8\textwidth]{images/#1}
  \caption{#2} 
  \label{fig:#1}
\end{figure}
}

\newcommand\subsecpar[1]{
    \subsection{#1}\paragraph{}
}
\newcommand\subsubsecpar[1]{
    \subsubsection{#1}\paragraph{}
}

\begin{document}
% cosas del overfull o nsq
\begin{sloppypar}

%título
\import{./sections}{title.tex}
\newpage

% hacer que aparezca la tabla de contenidos
\tableofcontents
\newpage

% hacer que aparezca la bibliografía

% para que la biblio salga sin hacer citaciones
\nocite{*}
\printbibliography[heading=bibintoc, title=Bibliografía]


\end{sloppypar}
\end{document}